%%
%% Introduction:
%% A paper about a method by using GeoHash to control the epidemic
%% Use covid-19 as example
%% Youwei Huang
%% IICT, CAS
\documentclass[sigplan,screen]{acmart}
\usepackage{subfigure}
\usepackage{gensymb}
\usepackage{algorithm}
\usepackage{algorithmicx}
\usepackage{algpseudocode}  
\usepackage{amsmath}
\renewcommand{\algorithmicrequire}{\textbf{Input:}}  
\renewcommand{\algorithmicensure}{\textbf{Output:}}  

\AtBeginDocument{%
  \providecommand\BibTeX{{%
    \normalfont B\kern-0.5em{\scshape i\kern-0.25em b}\kern-0.8em\TeX}}}

\begin{document}
\title{Epidemic Prevention and Control Based On GeoHash}
\author{Youwei Huang (Project Manager)}
\email{huangyw@iict.ac.cn}
\affiliation{%
	\institution{Institute of Intelligent Computing Technology, Chinese Academy of Sciences}
	\city{Suzhou}
	\state{Jiangsu}
	\country{China}
	\postcode{215000}
}
\author{Feng Lu (Algorithm Engineer)}
\email{lufeng20g@ict.ac.cn}
\affiliation{%
	\institution{Institute of Computing Technology, Chinese Academy of Sciences}
	\city{Beijing}
	\country{China}}
%% Abstract
\begin{abstract}
	COVID-19 (Coronavirus Disease 2019) which is a contagious disease caused by SARS-CoV-2\cite{hu2020characteristics} was first detected in Dec. 2019.
	Until 2021, this virus is still spreading around the world.
	Before the vaccine is widely vaccinated or the invention of specific medication, many measures have been taken by people to prevent the spread of the epidemic.
	In a special period, we have to quarantine the high-risk groups and lock down seriously infected regions.
	Here we propose a kind of dynamic block division technology based on \textbf{GeoHash} used to monitor, screen and control the epidemic areas.
	\textbf{GeoHash} is a public domain geocode system invented in 2008 by Gustavo Niemeyer\cite{niemeyer2008geohash}.
	We divide a map into several blocks and use \textbf{GeoHash} to encode the information of each block.
	Through the \textbf{GIS}, \textbf{GeoHash} can be easily decoded to original visual blocks on a digital map.
	A map generated by \textbf{GIS} is used for epidemic prevention and control, so it can be named \textbf{``Epidemic Map''}.
	Each block on such \textbf{Epidemic Map} contains the safety information and other important characteristics which are concerned by medical work.
	These dynamic blocks on the map can be scaled and represented as various regular geometric shapes.
	The vital information and the results of quantitative analysis of the data on each block support for decision-making, measures formulation, and effectiveness assessment of COVID-19 prevention and control.
	Such a kind of geographic information system can be used not only for preventing and controlling COVID-19 pandemic, but also be applicative in instances of other epidemic diseases.
\end{abstract}
\keywords{GeoHash, COVID-19, GIS, big data, epidemic prevention and control}
\begin{teaserfigure}
	\centering\includegraphics[width=\linewidth]{logo.png}
	\caption{GeoHash for geographic grid division}
	\Description{In 2008, Gustavo Niemeyer invented GeoHash which encoded a geographic location into a short string}
	\label{fig:teaser}
\end{teaserfigure}
%%
%% This command processes the author and affiliation and title
%% information and builds the first part of the formatted document.
\maketitle
\section{Introduction}
During the COVID-19 pandemic, many cities over the world were forced to lock down.
Wuhan City and the major cities in Hubei, China were put under lockdown on the 23rd and 24th of January, respectively\cite{lau2020positive}.
Lockdown meant the whole region was quarantined and cut off physical contact with the outside world.
The citizens were forbidden to leave their city or even their home.
The national medical team carried out centralized medical observation and treatment in quarantined cities.
Research shows, COVID-19 spread became weaker following lockdown\cite{lau2020positive}.
However the lockdown of a city can cause huge economic losses.
The lockdown of some vital areas can cause irreparable losses, such as financial center, political center, and industrial dependent cities.
Another issue that confuses people is how to distinguish whether the area they are in or where they are going is safe.
The current regional risk warning or lockdown is based on the administrative divisions as figure 2.
Cities, states or provinces all over the world have different sizes and irregular geographic borders.
There may be an outbreak in a city, but it does not mean that it spreads to all corners of the city.
On the contrary, there may be no epidemic in the center of neighboring cities, but there may already be a huge risk at the border with these surrounding cities.
Figure 2 is a map that shows the initial locked down cities in Wuhan province, China, a white block surrounded by red blocks is dangerous, even if it is not locked down.
\begin{figure}[htb]
	\centering
	\includegraphics[width=\linewidth]{hubei.png}
	\caption{Map of locked down administrative divisions of Hubei. [Public domain], via Wikipedia. (\url{https://en.wikipedia.org/wiki/COVID-19_lockdown_in_Hubei}).}
	\Description{Locked down cities in Hubei province, white blocks are unlocked and red blocks are locked}
\end{figure}
\\
We summarize the main weaknesses by using the current dividing measures:
\begin{enumerate}
	\item The border of a city is irregular and the transmission of virus doesn't follow the administrative division, so the prevention and lockdown can be not accurate.
	\item The administrative size of a city is fixed, but the disease is spreadable, so the region of lockdown can not be expanded flexibly.
	\item Due to regional differences, the information released in each region is not complete and not uniform.
	\item The news released by the local government can be lagging and users cannot get it in real time.
\end{enumerate}
Based on the above statements, it is not the best way to observe and control the epidemic area through the administrative division.
For infectious diseases, we have abandoned the common administrative methods. And we have adopted a technology based on GIS (Geographic Information System)\cite{clarke1986advances}. \textbf{GeoHash} is used to divide the map into several geometric blocks. The 2-dimensional geometric blocks on the map are encoded by GeoHash, they are reduced to 1-dimension and stored as string in any databases. These blocks are presented as regular geometry, but can be scaled according to the needs of different observation scope.
\\
Meanwhile, we do this technology is because it has the following advantages when controlling epidemic situation:
\begin{enumerate}
	\item The blocks divided by GeoHash are regular geometry, and the shape can be customized by observer.
	\item The blocks are generated dynamically and they are scalable according to the scope of infection.
	\item By combining with GIS, information about the epidemic situation can be encoded in GeoHash or directly saved.
	\item Block data can be easily quantified, as example of generating safety index.
	\item When such a GIS is released to the Internet, users get epidemic information in real time.
\end{enumerate}
In the practical and experimental scenarios, we use mobile application and web technology to develop such a particular GIS for medical prevention and treatment as shown in Figure 3. It works for medical workers as a visual auxiliary tool and share the results of epidemic data analysis. The \textbf{ASI} in Figure 3 is a value of ``Area Safety Index''. \textbf{ASI} represents the risk level of a region. The system contributes to enlighten and support decisions of governments, medical institutions, users, and other researchers who are doing the similar research with us.
\\
The core idea of this technology is dividing the earth to dynamic blocks. A block has the following characteristics:
\begin{enumerate}
	\item Blocks are regular geometry connected with each other. It can be understood as cellular grids.
	\item Blocks can be scaled on the map.
	\item Blocks are created only when they are meaningful.
	\item Scaling is limited, with the smallest and largest block size.
	\item Blocks scaling levels are discrete sizes, not continuous.
	\item A block stores structured data, which can be used for computing and analysing various attributes.
\end{enumerate}
The last item in the above list indicates that the purpose of dividing by blocks is to perform quantitative analysis related to the epidemic situation.
\\
We will also introduce other related work in controlling epidemic by using similar information technology and computer visualization technology.
Then we will focus on how we use \textbf{GeoHash} to divide blocks on the map, and explain the methods of quantifing epidemic data to provide area risk warning.
Finally, by using our GIS example, we will give our experimental and test results, and summarize the conclusion.
\begin{figure*}[hptb]
	\centering
	\subfigure[larger scale]{
		\includegraphics[width=\textwidth]{geogrids1.png}
	}
	\subfigure[smaller scale]{
		\includegraphics[width=\textwidth]{geogrids2.png}
	}
	\caption{Dynamic blocks with ASI in geographic information system}
\end{figure*}
\section{Related Work}
There are some mature cases of controlling and treating COVID-19 pandemic by using information technology and Internet data.
Many studies on COVID-19 have recently emerged, and various data science applications combating the pandemic have been reported recently\cite{latif2020leveraging}.
The main functions of these systems or softwares are listed as follows:\cite{jia2020big}
\begin{itemize}
	\item Tracking of people's movements.
	\item Early warning of high-risk areas.
	\item Screening of asymptomatic potential infections.
	\item Drug development.
	\item Information release and policy support.
\end{itemize}
\subsection{Data Visualization Analysis}
The computer can visualize all kinds of structured data and convey the visual information to users.
Visualization technology presents data to users by drawing charts and graphics, in which the data is represented by symbols, such as bar charts, line charts, pie charts, maps and etc\cite{jensen1992harvard}.
\\
Figure 4 shows the global COVID-19 epidemic situation in the form of map charts. The epidemic maps are updated by WHO (World Health Organization)\footnote{https://www.who.int} in real time, to display the number of cases around the world.
\begin{figure}[htb]
	\centering
	\subfigure[Choropleth Map]{
		\includegraphics[width=\linewidth]{worldmap1.png}
	}
	\subfigure[Bubble Map]{
		\includegraphics[width=\linewidth]{worldmap2.png}
	}
	\caption{WHO coronavirus disease dashboard. [Public domain], via WHO (\url{https://covid19.who.int})}
	\Description{The figures are captured from the official website of WHO, the web page uses a dashboard to show the COVID-19 situation over the world.}
\end{figure}
\\
Maps in Figure 4 uses two styles of presentation: choropleth and bubble.
One uses the depth of color to show the severity of the epidemic situation in each country, and another uses the bubble size to show the number of infections.
No matter what kind of map, its role is to help the local people easy to understand the severity of the epidemic, and prompt the local government to take actions to treat and control the epidemic situation.
These two figures (Figure 4) are similar to the previous Figure 2, except that Figure 2 only shows the data of one province.
Contents in these maps include like: confirmed cases, deaths, historical cases, added cases, and regional lockdown status.
Some disadvantages of such kind of maps are discussed in the introduction section (Section 1).
\\
The bar charts and line charts are aslo widely used in epidemic data analysis.
These graphs are mostly used to show the trend of epidemic situation and transmission cycle.
Figure 5 is an example that uses both bar chart and line chart to perform the trend of daily new cases from Jan. 2020 to Jan. 2021 in the US. The red line in this chart is the 7-day moving average curve.
Other trends from different types of data, such as death trends, can be found on the official CDC\footnote{https://covid.cdc.gov/covid-data-tracker} website.
\begin{figure}[htb]
	\centering\includegraphics[width=\linewidth]{linebar-us-1-15.png}
	\caption{Daily trends in number of COVID-19 new cases in the US reported to CDC. [Public domain], via CDC (\url{https://covid.cdc.gov/covid-data-tracker/\#trends_dailytrendscases})}
	\Description{The figure is a screenshot from CDC official web page, it shows the daily new cases in the US by drawing a bar and line chart.}
\end{figure}
\\
The line charts and bar charts reflect historical epidemic data from time series, while map charts reflect epidemic data from spatial distribution.
In other related work survey, visual data analysis are used to study the relationship between population mobility and the epidemic spreading pattern.
During the early outbreak of coronavirus in Wuhan, China, the graphs in a research suggested that the number of confirmed cases in other provinces were directly proportional to the inflow of Wuhan population.
The research group also used the pattern derived from the data analysis to predict the number of infections.\cite{chen2020data}
\subsection{Geographic Information System}
The maps we described in the previous section are charts, the information carried by map charts is limited, unflexible, not automatic analysis and not real-time.
Although charts can provide visual perception, users need to analyze the graphs by themselves, but GIS can integrate analysis, prediction and other practical functions.
A GIS (geographic information system) is a conceptualized framework that provides the ability to capture and analyze spatial and geographic data\cite{clarke1986advances}.
Since the outbreak of the epidemic, a number of geographic information systems have been built or have added real-time epidemic related functions, such as ``epidemic map displays'', ``fever clinic queries'' and ``passenger information queries''.
Based on existing commercial GIS softwares, they made important contributions to epidemic prevention and control\cite{zhou2020covid}.
Some of the systems have the function of dynamic zoom map, which can display the situations of different scaling areas, from state level to community level.
For example, Figure 6 shows a map with an epidemic layer on it, it is a GIS tool that shows critical information about COVID-19 cases in an area so the users can make more informed decisions about where to go.
\begin{figure}[htb]
	\centering
	\subfigure[State Level]{
		\includegraphics[width=0.45\linewidth,height=80mm]{state-level.png}
	}
	\subfigure[County Level]{
		\includegraphics[width=0.45\linewidth,height=80mm]{county-level.png}
	}
	\caption{COVID layer in Google Map of different scales}
\end{figure}
\\
In Figure 6, \textbf{Google Map} adds a COVID-19 layer to the GIS, it also quantities an safety index by using the 7-day average for the number of new cases per 100,000 people.
It also indicates whether cases are increasing or decreasing.
The layer's colors indicate:\footnote{https://support.google.com/maps/answer/9795160}
\begin{itemize}
	\item Grey: Less than 1 case
	\item Yellow: 1-10 cases
	\item Orange: 10-20 cases
	\item Dark orange: 20-30 cases
	\item Red: 30-40 cases
	\item Dark red: 40+ cases
\end{itemize}
Unlike \textbf{Google Map}, we perform the quantitative analysis based on dynamic GeoHash blocks instead of administrative regions.
The safety quantification algorithm we use not only includes the simple infected cases.
In the following chapters, I will focus on our research work about how to use GeoHash to improve the GIS in epidemic prevention and control.

\section{Pre-Work}
Our challenges are mainly from two aspects: data collection and data quantification.
Data collection is integrated in GIS, which requires users' devices to upload the current positions of the users, and the server needs to store the position data submitted by users.
Quantitative analysis of information is a more complex process and our work is to quantify the collected messy data in GeoHash blocks.
\\
For data collection, the system can provide a mobile application for users to view the surrounding epidemic situation.
At the same time, in order to obtain the surrounding epidemic safety situation, users need to upload their own GPS information.
Under the privacy policy, we only collect users' GPS information, but we will not save users' personal informationa. So we don't track or monitor users' positions in the system.
This method of collection is carried out anonymously, and each record is stored as a virtual and unknown identity which means only users know their own locations but other users cannot get it.
\\
After the diagnosis, the medical workers mark the confirmed cases through their virtual identities.
Other users don't know these users' real identity, but they can browse the surrounding infection.
The medical workers know their patients' real identities, but they cannot track their location information without user authorization.
To be simple, our solution is to use the virtual identity or encode user information to separate location information and real user information.
This is an approach to keep a balance between keeping user privacy and collecting data for epidemic control.
\\
For example, in China, a kind of QR code carried by users called \textbf{Health Code} shows one's health status but doesn't tell one's information.
Each QR code corresponds to a unique user in the real world.
In the real world, people use the health code to know the health status of themselves or others.
\\
For data quantification, we need to divide the map to blocks first by using GeoHash.
At the step of GPS data collection, we get the longitude and latitude values from the user.
GeoHash will calculate which block the user belongs to.
Suppose a GeoHash block as a set $B_1$ that includes many users, we quantify the safety index of $B_1$ based on confirmed cases and the total number of the user.
The historical users, ``visitors'' in a block also need to be considered.
These visitors who are diagnosed in the short term will have negative effects on a block.
As discussed above (in Section 2.1), a research has proposed that the number of people infected in a region is positively correlated with the inflow of the population from high-risk areas.
That means, when collecting position data, we noy only need to save user current position, but also need to save user historical location data.
The general storage and calculation process is shown in Figure 7, and the detailed methods and algorithms are described in Section 4.
\begin{figure}[htb]
	\centering\includegraphics[width=\linewidth]{process.pdf}
	\caption{The process of data collection and quantification}
\end{figure}
\section{Implementation Methods}
In Pre-Work section, user position information is obtained through the mobile terminal.
The information contains some necessary data: userid, longitude and latitude, and \textbf{USI}\footnote{USI is ``User Safety Index'', a value of user's health in our system}.
Based on the above data, we can divide users into different blocks.
\subsection{Grid Division}
Our first step is to divide the earth into blocks or grids. GeoHash technology can help us achieve this step.
GeoHash can perform the following operations:
\begin{itemize}
	\item GeoHash divides a two-dimensional map into buckets of grid according to the range of longitude and latitude.
	\item GeoHash encodes a geographic location (longitude and latitude) into a string of binary codes.
	\item Through the incremental operation of encoded binary code, these grids are chained to a Z-order curve as Figure 8.
\end{itemize}
How does Geohash achieve the above operations?
\\
We know that the geographical range of longitude is from -180\degree\ to 180\degree\ and latitude is from -90\degree\ to 90\degree\ \cite{crossley1999guide}.
We can divide the longitude into two intervals: [-180\degree, 0\degree], [0\degree, 180\degree].
We denote the two intervals by binary number ``0'' and ``1'' respectively.
In an equivalent way, we divide the latitude into two intervals: [-90\degree, 0\degree], [0\degree, 90\degree] and denote them by ``0'' and ``1''.
\\
Then we can use ``00'', ``01'', ``10'', ``11'' to represent the four grids.
Figure 8 gives examples of ``divide the earth into 4 grids'' and ``divide the earth into 16 grids''.
But in this way, the area of each grid is very large, and the grid precision is too low.
The same way is used to further subdivide into 16, 64 grids, etc.
We get higher precision by adding the bits of GeoHash binary which means the earth is divided into more and smaller grids.
\begin{figure}[htb]
	\centering
	\subfigure[4 Grids]{
		\includegraphics[width=0.45\linewidth,height=33mm]{4-grids.png}
	}
	\subfigure[16 Grids]{
		\includegraphics[width=0.45\linewidth,height=33mm]{16-grids.png}
	}
	\caption{Z-order curve to show GeoHash grids division}
\end{figure}
\subsection{Gird Size}
%%In addition, we found that these grids can be strung to an z-order curve, which means we can store the two-dimensional map grids in a one-dimensional array.
%%Any user shall belong to a grid.
The following Table 1 contrasts the geographical length and GeoHash bits length at around 30\degree\ latitude.
\begin{table}[htb]
	\caption{Comparison table of GeoHash bits and estimated geographical length of one grid (around 30\degree\ latitude)}
	\begin{tabular}{lll}
		\toprule
		East-West Length(m) & South-North Length(m) & Bits \\
		\midrule
		32.67               & 19.05                 & 20*2 \\
		65.34               & 38.1                  & 19*2 \\
		130.68              & 76.2                  & 18*2 \\
		261.36              & 152.4                 & 17*2 \\
		522.72              & 304.8                 & 16*2 \\
		1045.44             & 609.6                 & 15*2 \\
		2090.88             & 1219.2                & 14*2 \\
		4181.76             & 2438.4                & 13*2 \\
		8363.52             & 4876.8                & 12*2 \\
		16727.04            & 9753.6                & 11*2 \\
		33454.08            & 19507.2               & 10*2 \\
		66908.16            & 39014.4               & 9*2  \\
		133816.32           & 78028.8               & 8*2  \\
		267632.64           & 156057.6              & 7*2  \\
		535265.28           & 312115.2              & 6*2  \\
		1070530.56          & 624230.4              & 5*2  \\
		2141061.12          & 1248460.8             & 4*2  \\
		4282122.24          & 2496921.6             & 3*2  \\
		8564244.48          & 4993843.2             & 2*2  \\
		17128488.96         & 9987686.4             & 1*2  \\
		\bottomrule
	\end{tabular}
\end{table}
Since the geometric shape of the earth is a sphere, even with the same GeoHash bit, the grids of high latitude contains less geographical length than those of low latitude.
Attention: the length data in Table 1 is generated from the grids around 30\degree\ latitude, it doesn't represent all the grids or an average value.
\\
Use the following formulas (1) and (2) to estimate the length of the two edges of a grid and it assumes the earth is completely spherical.
We use $G_{NS}$ to represent the length of north-south edge and $G_{EW}$ to represent the length of east-west edge as they are stroked in Figure 9.
The index $geobit$ is the length of total GeoHash bits which is used to represent a grid.
The length of a meridian which is $L_{meridian}$ in Formula (1) has been estimated at 20,003.93 km (12,429.9 miles) on a modern ellipsoid model of the earth (WGS 84)\cite{weintrit2013so}.
The length of the equator which is $L_{equator}$ in Formula (2) is about 40,075 km (24,901 miles) long\cite{equator2011}.
Formula (1) is used to estimate the length of the north-south edge of a grid.
Formula (2) is used to estimate the length of the east-west edge of a grid at latitude $\varphi$.
\begin{equation}
	G_{NS}\approx\frac{L_{meridian}}{2^{geobit/2}}
\end{equation}
\begin{equation}
	G_{EW}(\varphi)\approx\frac{L_{equator}\times\cos(\varphi)}{2^{geobit/2}}
\end{equation}
\\
The reason why the word ``estimate'' is used when caculating $G_{NS}$ and $G_{EW}$ here is that the earth is actually elliptical.
Earth ellipsoid will cause the following two issues:
\begin{itemize}
	\item The meridian and equator are different in length, grids at different latitudes own different $G_{NS}$ values.
	\item If the precision of the grid is not high enough, the grid is not a rectange, two different $G_{EW}$ values will appear in one grid.
\end{itemize}
In fact, under the same GeoHash bits, whether it is ``Earth Ellipsoid'' or ``Earth Spheroid'', the length of the two edges of a grid depends only on latitude.
If we need the accurate length of the two edges, we should start a deep discussion about ``Length of a degree of longitude'' and ``Length of a degree of latitude'', which are beyond the research scope of this paper.
Here we list our final calculation formulas when the earth is modelled by an ellipsoid, this is much more complicated but accurate than the spherical earth.
\begin{equation}
	a=\frac{L_{equator}}{2\pi}\quad b=\frac{L_{meridian}}{2\pi}
\end{equation}
\begin{equation}
	f=\frac{a-b}{a}\quad e^2=f(2-f)
\end{equation}
\begin{equation}
	\bigtriangleup\varphi=\frac{180}{2^{geobit/2}}
\end{equation}
\begin{equation}
	G_{EW}(k)=\frac{2\pi a\cos(k\Delta\varphi)}{2^{geobit/2}\sqrt{1-e^2\sin^2k\Delta\varphi}}
\end{equation}
\begin{equation}
	G_{NS}(k)=a(1-e^2)\int_{k\Delta\varphi}^{(k+1)\Delta\varphi}\frac{\operatorname d\phi}{{(1-e^2\sin^2\phi)}^\frac32}
\end{equation}
\begin{equation}
	k\in\{-2^{geobit/2-1}, -2^{geobit/2-1}+1, \ldots, 2^{geobit/2-1}-1\}
\end{equation}
\begin{figure}[htb]
	\centering\includegraphics[width=\linewidth]{earth.png}
	\caption{Grid of elliptical earth}
\end{figure}
\\
Formula (3), (4), (5), (6), (7) and (8), give the methods to estimate the grid size of eplliptical earth.
They are used to calculate the length of each side of a grid.
The formulas are derived from \textit{The Mercator Projections}\cite{osborne2013mercator}.
Any GeoHash grid in our GIS has 3 different side lengths, which are represented by $G_{NS}(k)$, $G_{EW}(k)$ and $G_{EW}(k+1)$ respectively.

\subsection{Block Storage}
We store all these blocks and user data in common databases on the server.
Data storage includes two main parts:
\begin{enumerate}
    \item user historical location data
    \item block data
\end{enumerate}
One block contains many users, and one user has many position logs.
Block and user information are bidirectional associated, which is called “many to many” in the field of computer data storage model.
This section will discuss the data storage models and algorithms.
\\
Before storing the block, we have to encode the position by GeoHash.
The first step is to transform longitude and latitude to binary code.
Referring to the description about grid division in Section 4.1, we adopt the following Algorithm 1.
\begin{algorithm}[htb]
    \caption{Transform position to GeoHash bit}
    \begin{algorithmic}[1]
        \Require
            User's $longitude$ value, the data type is double.
            User's $latitude$ value, the data type is double.
            The length of GeoHash $bit$, for longitude or latitude, so the full length is $2\times bit$.
        \Ensure
            The result of GeoHash (binary code), the type is string or integer array with value 0 or 1.
        \Function {geohash}{$longitude,latitude,bit$}
            \State $result=[]$
            \State $i=0$
            \State $lat[0]=-90$
            \State $lat[1]=90$
            \State $lon[0]=-180$
            \State $lon[1]=180$
            \If{$lon[0]\le longitude\le lon[1]$ \textbf{and} $lat[0]\le latitude\le lat[1]$}
                \While{$i<bit*2$}
                    \State $mid=(lon[0]+lon[1])/2$
                    \If{$mid>longitude$}
                        \State $result[i++]=0$
                        \State $lon[1]=mid$
                    \Else
                        \State $result[i++]=1$
                        \State $lon[0]=mid$
                    \EndIf

                    \State $mid=(lat[0]+lat[1])/2$
                    \If{$mid>latitude$}
                        \State $result[i++]=0$
                        \State $lat[1]=mid$
                    \Else
                        \State $result[i++]=1$
                        \State $lat[0]=mid$
                    \EndIf
                \EndWhile
            \EndIf
            \State \Return $result$
        \EndFunction
    \end{algorithmic}
\end{algorithm}
\\
In the section 4.1 we have discussed that a GeoHash binary code can be divided into two parts: longitude part and latitude part, and they have the same number of bits in a GeoHash binary code.
In Algorithm 1, longitude and latitude are rankded one by one.
For example: in a returned GeoHash binary code, the first bit represents the range of longitude, then the second bit represents the range of latitude, and the third bit is longitude again, and so on.
This is for code simplicity, theoretically, as long as the encoding and decoding rules are consistent, how to sort GeoHash binary codes is not the most important thing here.
\\
If we directly save the GeoHash binary code of a block, a string of binary code is too long.
In the practical production environment, we do convert binary code to hexadecimal code and store the "hex code" in the non-relational database.
Therefore, a string of hex code can represent a block, and also it contains a range of positions.
Figure 10 shows the complete process of "From position to GeoHash hex code".
\\
When a position needs to be determined, hexadecimal code can be easily converted to binary code, and then the binary code can be decoded to the range of longitude and latitude by inversing GeoHash algorithm.
The accuracy of restored position depends on the number of GeoHash bits.
In this project, we choose 10-20 bits GeoHash to encode longitude and latitude respectively, so the total length of a generated GeoHash binary is 20-40 bits.
In the practical test, we consider such sizes of block are the most reasonable observation range for medical workers.
Thanks to the flexibility of the algorithm, we can adjust it according to the requirements.
\\
Figure 10 has illustrated the storage data structure in the non-relational database.
Hashes are maps between string keys and string values.
We set the user IDs as keys and user health information as values in a hash table.
In the database, one block corresponds to one hash table, and the hexadecimal code of block is the name of hash table.
\begin{figure}[htb]
	\centering\includegraphics[width=\linewidth]{block-storage.png}
	\caption{The process of block storage}
\end{figure}
\\
In this way, when a user enters a block, a user ID and information as a "key-value" pair will be added to the hash table.
When the user changes the position and enters another block, the user will be moved out of the current hash table and then into the hash table of the new block.
\\
For those blocks that there are no user in it, we call them \textbf{"Depopulated Blocks"}.
These blocks are logically existed, but they are not physically stored.
By clearing the \textbf{"Depopulated Blocks"}, we free the storage space.
In contrast, when a user appears in a "depopulated block", a new block will be created dynamically in the storage space, which means a new related hash table will be generated, and the user's information will be saved in the table.
\\
For the blocks that the user has entered historically, these block logs can be recorded in a relational database table at the row level.
The users' IDs are the foreign keys in the table.
When these block logs are queried from the database, as Figure 11, these blocks are organized as nodes in a link or an array following the user.
We call them \textbf{``User GeoHash Block Chain''}, \textbf{UGBC} for short.
For a block node in a link, the closer to the current time the user has entered, the more ahead in this link, the blocks that the user entered ealier will be placed at the tail of the link.
\begin{figure}[htb]
	\centering\includegraphics[width=\linewidth]{block-log.png}
	\caption{User GeoHash Block Chain}
\end{figure}
\\
COVID-19 requires us to query the past 14-days records which will generate a 14-days \textbf{UGBC}.

\subsection{Grid Quantification}
\section{Test}
\section{Conclusions}
%%
%% The next two lines define the bibliography style to be used, and
%% the bibliography file.
\bibliographystyle{unsrt}
\bibliographystyle{ACM-Reference-Format}
\bibliography{refs}
\end{document}
\endinput
